\linespread{1.15}\selectfont

%%%%%%%%%%%%%%%%%%%%%%%%%%%%%%%%%%%%%%%%%%%%%%%%%%%%%%%%%%%%%%%%%%
%%%%%%%%%%%%%%%%%%%%%%%%%%%%%%%%%%%%%%%%%%%%%%%%%%%%%%%%%%%%%%%%%%
\chapter{Основные определения временных рядов}
%%%%%%%%%%%%%%%%%%%%%%%%%%%%%%%%%%%%%%%%%%%%%%%%%%%%%%%%%%%%%%%%%%
%%%%%%%%%%%%%%%%%%%%%%%%%%%%%%%%%%%%%%%%%%%%%%%%%%%%%%%%%%%%%%%%%%

%%%%%%%%%%%%%%%%%%%%%%%%%%%%%%%%%%%%%%%%%%%%%%%%%%%%%%%%%%%%%%%%%%
\section{Определение и свойства временного ряда}
%%%%%%%%%%%%%%%%%%%%%%%%%%%%%%%%%%%%%%%%%%%%%%%%%%%%%%%%%%%%%%%%%%
Временной ряд определяется как последовательность значений, упорядоченных во времени и характеризующих уровень состояния и изменения наблюдаемого показателя.
Особенностью временного ряда является то, что каждое значение показателя зависит от прошлых состояний, т.е. важен порядок следования значений .

Временные ряды различаются по следующим признакам:
\begin{enumerate}
	\item \textit{Длина ряда}. Здесь употребляется в смысле числа наблюдений параметра, хотя может означать и время, прошедшее от начального до конечного наблюдения. Кендел в \cite{kendel} говорит, что в отличии от обычной статистической работы, в анализе временных рядов количество информации не является пропорциональным числу членов выборки, т.к. <<...последовательные величины не являются независимыми>>.
	\item \textit{Дискретность и непрерывность}. Определяется характером изменения времени,  в течение которого производится наблюдение. Согласно дискретные ряды могут быть получены из непрерывных двумя способами:
	\begin{itemize}
		\item выборкой из непрерывных временных рядов через определенный интервал;
		\item накоплением значения в течение некоторого периода времени.	
	\end{itemize} 
	\item \textit{Детерминированность}. Если будущие значения временного ряда определены какой-либо математической
 функцией, тогда временной ряд называется детерминированным. Если же будущие значения описываются только с помощью
 распределения вероятностей, то временной ряд называется недетерминированным, случайным или стохастическим.
 В свою очередь стохастический временной ряд может быть стационарным или нестационарным. Ряд называется стационарным, если его свойства не зависят от начала отсчета времени. <<В частности, он имеет постоянное математическое ожидание (т.е. среднее
  значение, относительно которого он варьирует), постоянную дисперсию, определяющую размах его колебаний относительно
   среднего значения, а также постоянную автоковариацию и коэффициенты автокорреляции>>.
 Чтобы дискретный ряд был строго стационарным, дисперсия любой совокупности наблюдений не должно изменяться при сдвиге
 времени наблюдения на любое целое число.
	\item \textit{Моментные и интервальные} временные ряды. Интервальный ряд - последовательность, в которой уровень
явления относят к результату, накопленному за определенный интервал времени. Если же уровень
ряда характеризует изучаемое явление в конкретный момент времени, то совокупность уровней образует моментный ряд.
 Важное отличие моментных рядов от интервальных состоит в том, что сумма уровней интервального ряда дает вполне
 реальный показатель --  общее значение за интервал.
	\item \textit{Полные и неполные} временные ряды. Полные ряды имеют место, когда даты регистрации или окончания 
периодов следуют друг за другом с равными интервалами, неполные - когда принцип равных интервалов не 
 соблюдается.
\end{enumerate}


\section{Компоненты временного ряда}
В изучении временных рядов большое место занимает вопрос о закономерностях их движения на протяжении длительного периода. Выявление зависимостей, действующих во времени является сложной процедурой, поскольку данные зависимости формируются под действием многих факторов. Выделяется две группы факторов:
\begin{itemize}
	\item определяющие основную тенденцию динамики;
	\item вызывающие случайные колебания.
\end{itemize}

Тогда по временной ряд можно представить в следующем виде:
$$
x_t = \xi_t + \epsilon_t
$$
где, $\epsilon_t$ генерируется случайным неавтокоррелированным процессом с нулевым математическим ожиданием
и конечной (не обязательно постоянной) дисперсией, а величина $\xi_t$ может быть генерирована либо детерминированной функцией, либо случайным процессом, либо какой-нибудь их комбинацией. Величины $\epsilon_t$ и $\xi_t$ различаются характером воздействия на значения последующих членов ряда. Переменная $\epsilon_t$ влияет только на значение синхронного ей члена ряда, в то время как величина $\xi_t$ в известной степени определяет значение нескольких или всех последующих членов ряда. 
Основной тенденцией, или трендом, называется характеристика процесса изменения явления за длительное время, освобожденная от случайных колебаний, создаваемых второй группой факторов.  В модели тренд обозначается через $\xi_t$ и может быть выражен как детерминированной так и случайной функциями или их комбинацией. 
Колеблемостью следует называть отклонения уровней отдельных периодов времени от тенденции динамики (тренда).

Однако можно наблюдать иерархию тенденций и колебаний: та величина, которая для, например столетия, выступает как колебания, на интервале времени низшего порядка, например трех-пяти лет, может выступать как тенденция.

Временной ряд следует рассматривать как смесь четырех компонент:
\begin{itemize}
	\item тренда,
	\item регулярных колебаний относительно тренда,
	\item сезонной компоненты,
	\item остатка или несистематического случайного эффекта.
\end{itemize}

В общем случае временной ряд представляется в виде суммы этих 4-х компонент. Одной из задач анализа временных рядов является разложение ряда на составляющие его компоненты с целью их изучения. Компоненты временного ряда ненаблюдаемы. Они являются теоретическими величинами. Их выделение и составляет предмет анализа временного ряда. 

\section{Классификация задач моделирования и анализа временных рядов}
Основным средством анализа и прогноза временного ряда является модель. Модели временных рядов удобно применять, в частности, для дискретных систем <<...т.е. таких систем, в которых возможность произвести наблюдение и предпринять регулирующие действия возникает через равные интервалы времени времени>>.
Понятие модель используется в двух значениях: как модель временного ряда, выражающая закон
генерирования членов ряда, и как прогнозная модель. Главное отличие этих двух типов моделей в том,
что на выходе модели временного ряда фактические члены ряда, а на выходе прогнозной модели — оценки будущих
членов ряда.
Также выделяются следующие важные прикладные области применения моделей временных рядов:
\begin{enumerate}
	\item Прогнозирование будущих значений временного ряда по его текущим и прошлым значениям.
	\item Определение передаточной функции системы.
	\item Проектирование простых регулирующих схем с прямой и обратной связями.
\end{enumerate}
Выделяются следующие цели анализа временных рядов:
\begin{enumerate}
	\item \label{l1} Построение системы математического вида, которая описывает поведение временного ряда в сжатом виде.
	\item \label{l2} Для объяснения поведения временного ряда с помощью других переменных в качестве гипотезы строится модель.
	\item \label{l3} Результаты анализа, полученные в  \ref{l1} или  \ref{l2} могут быть использованы для прогнозирования поведения ряда.
	\item \label{l4} В случае \ref{l2} возможен контроль системы путем выработки сигналов о наступающих изменениях или путем исследования того, что может случиться, если изменить некоторые из параметров модели.
	\item \label{l5} Анализ совместного развития во времени нескольких переменных.
\end{enumerate}

Объединяя эти точки зрения можно выделить следующие задачи, которые могут быть решены при моделировании временных рядов:
\begin{enumerate}
\item Построение формализованного представления моделируемой системы с выделением ее значимых параметров - определение природы временного ряда.
\item Прогнозирование будущих значений временного ряда, т.е. определение вида передаточной функции.
\end{enumerate}
Прогнозирование -- это научное, основанное на системе установленных причинно-следственных связей и закономерностей, выявление состояния и вероятностных путей развития явлений и процессов.

Выделяются следующие типы прогнозов:
\begin{enumerate}
\item В зависимости от целей исследования прогнозы делятся на поисковые и нормативные:
	\begin{itemize}
	\item Нормативный прогноз – это прогноз, который предназначен для указания возможных путей и сроков достижения заданного, желаемого конечного состояния прогнозируемого объекта, то есть нормативный прогноз разрабатывается на базе заранее определенных целей и задач.
	\item Поисковый прогноз не ориентируется на заданную цель, а рассматривает
возможные направления будущего развития прогнозируемого объекта, то есть выявление того, как будет развиваться объект в будущем полностью зависит от сохранения существующих тенденций.
	\end{itemize}
\item В зависимости от специфики области применения прогноза
и от объекта прогнозирования прогнозы подразделяются на:
	\begin{itemize}
		\item естественноведческие – это прогнозы в области биологии, медицины и так далее;
		\item научно-технические - это, например, инженерное прогнозирование технических характеристик узлов, деталей и 	так далее.
		\end{itemize}
\item В зависимости от масштабности объекта, прогнозы бывают:
	\begin{itemize}
	\item глобальные – рассматривают наиболее общие тенденции и закономерности в мировом масштабе;
	\item макроэкономические – анализируют наиболее общие тенденции явлений и процессов в масштабе экономики страны в целом;
	\item структурные  (межотраслевые и межрегиональные) – предсказывают развитие экономики в разрезе отраслей;
	\item региональные – предсказывают развитие отдельных регионов;
	\item отраслевые – прогнозируют развитие отраслей;
	\item микроэкономические – предсказывают развитие отдельных предприятий, производств и так далее.
	\end{itemize}
\item По сложности прогнозы различают:
	\begin{itemize}
		\item сверхпростые – прогноз на основе одномерных временных рядов, когда отсутствуют связи между признаками;
		\item простые – прогнозы, предполагающие учет оценки связей между факторными признаками;
		\item сложные – прогнозы, оценка связей между признаками в которых определяется на основе системы уравнений или многофакторного динамического прогнозирования.
		\end{itemize}
\end{enumerate}

Модели основаны на допущении о том, что основные факторы и тенденции прошлого периода сохранятся и на период прогноза, или что направление и изменение тенденций в рассматриваемой перспективе можно обосновать и учесть, т.е. предполагается большая инерционность систем. 

