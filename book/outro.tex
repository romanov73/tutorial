\linespread{1.15}\selectfont

%%%%%%%%%%%%%%%%%%%%%%%%%%%%%%%%%%%%%%%%%%%%%%%%%%%%%%%%%%%%%%%%%%
%%%%%%%%%%%%%%%%%%%%%%%%%%%%%%%%%%%%%%%%%%%%%%%%%%%%%%%%%%%%%%%%%%
\chapter*{Заключение}\addcontentsline{toc}{chapter}{Заключение}
%%%%%%%%%%%%%%%%%%%%%%%%%%%%%%%%%%%%%%%%%%%%%%%%%%%%%%%%%%%%%%%%%%
%%%%%%%%%%%%%%%%%%%%%%%%%%%%%%%%%%%%%%%%%%%%%%%%%%%%%%%%%%%%%%%%%%

Целью данного пособия является фоpмиpование у студентов знаний об аппаратном обеспечении и архитектуре вычислительных систем, принципах построения и функционирования основных устройств данных систем. Представлены основные этапы развития, семейства и типы вычислительных систем.

Также дается пояснение каким образом программы, написанные на языках программирования высокого уровня, выполняются вычислительными системами. Рассмотрены основные уровни архитектуры вычислительных систем:
\begin{itemize}
	\item цифровой логический уровень;
	\item уровень микроархитектуры;
	\item уровень архитектуры набора команд;
	\item уровень операционной системы;
	\item уровень ассемблера.
\end{itemize}
Описаны функциональные особенности и механизмы взаимодействия данных уровней.

Более подробную информацию по вопросам, рассмотренным в данном пособии, можно найти в \cite{Gook}, \cite{Kalabekov}, \cite{Lehin}, \cite{Tanenbaum}, и \cite{Asmworld}.