\thispagestyle{empty}

\noindent
ББК 32.973-04я73\\
УДК 004.3(075.8)

\hangindent=0.7cm \hangafter=0 \noindent
А 76
\vspace{0.8cm}

\hangindent=3.5cm \hangafter=0 \noindent
Рецензенты:\\
доктор технических наук, профессор И. В. Семушин\\
кафедра <<Информационные технологии>> Ульяновского государственного университета

\vspace{0.6cm}
\begin{normalsize}
\noindent
Утверждено редакционно-издательским советом университета в качестве учебного пособия
\end{normalsize}

\vspace{0.6cm}
\noindent

ИА : учебное пособие~/
составители В.~В.~Воронина, А.~А.~Романов. -- Ульяновск: УлГТУ, 2015. -- 94 с.

ISBN ??????????

\begin{normalsize}
В рамках данного пособия рассматривается одно из направлений обработки данных - Data Mining,
а именно интеллектуальный анализ временных рядов. В последние годы к этой тематике проявляется большой
интерес поскольку с использованием временных рядов возможно моделировать большое количество самых 
разнообразных явлений и процессов, которые являются источником рядов. Построенные модели временных рядов
призваны выявлять структурные особенности исследуемых процессов с целью их анализа и прогнозирования состояния.
Прогноз временного ряда используется для повышения эффективности принятия решений.

Пособие предназначено для студентов направления 09.03.04 <<Программная инженерия>>, 
изучающих дисциплину <<Интеллектуальный анализ данных>>.
\end{normalsize}

\vspace{0.6cm}
\begin{flushright}
\textbf{УДК 004.3(075.8)}\\
\textbf{ББК 32.973-04я73}
\end{flushright}

\vfill

\noindent
\begin{minipage}{0.4\textwidth}
	ISBN ??????????
\end{minipage}
\hfill
\begin{minipage}{0.55\textwidth}
	\begin{flushright}
		\textcopyright{} Составление. Воронина В. В.,\\ Романов А. А., 2015\\
		\textcopyright{} Оформление. УлГТУ, 2015
	\end{flushright}
\end{minipage}