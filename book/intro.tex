%%%%%%%%%%%%%%%%%%%%%%%%%%%%%%%%%%%%%%%%%%%%%%%%%%%%%%%%%%%%%%%%%%
%%%%%%%%%%%%%%%%%%%%%%%%%%%%%%%%%%%%%%%%%%%%%%%%%%%%%%%%%%%%%%%%%%
\chapter*{ПРЕДИСЛОВИЕ}\addcontentsline{toc}{chapter}{Предисловие}
%%%%%%%%%%%%%%%%%%%%%%%%%%%%%%%%%%%%%%%%%%%%%%%%%%%%%%%%%%%%%%%%%%
%%%%%%%%%%%%%%%%%%%%%%%%%%%%%%%%%%%%%%%%%%%%%%%%%%%%%%%%%%%%%%%%%%

В данном пособии представлено описание подхода к разработке программной системы, предназначенной для моделирования временных рядов. Освещены особенности построения научно-исследовательского программного обеспечения.

В главе 1 представлено...

Пособие предназначено для студентов направления 09.03.04 <<Программная инженерия>>, 
изучающих дисциплину <<Интеллектуальный анализ данных>>
